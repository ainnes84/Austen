\documentclass{article}\usepackage[]{graphicx}\usepackage[]{color}
%% maxwidth is the original width if it is less than linewidth
%% otherwise use linewidth (to make sure the graphics do not exceed the margin)
\makeatletter
\def\maxwidth{ %
  \ifdim\Gin@nat@width>\linewidth
    \linewidth
  \else
    \Gin@nat@width
  \fi
}
\makeatother

\definecolor{fgcolor}{rgb}{0.345, 0.345, 0.345}
\newcommand{\hlnum}[1]{\textcolor[rgb]{0.686,0.059,0.569}{#1}}%
\newcommand{\hlstr}[1]{\textcolor[rgb]{0.192,0.494,0.8}{#1}}%
\newcommand{\hlcom}[1]{\textcolor[rgb]{0.678,0.584,0.686}{\textit{#1}}}%
\newcommand{\hlopt}[1]{\textcolor[rgb]{0,0,0}{#1}}%
\newcommand{\hlstd}[1]{\textcolor[rgb]{0.345,0.345,0.345}{#1}}%
\newcommand{\hlkwa}[1]{\textcolor[rgb]{0.161,0.373,0.58}{\textbf{#1}}}%
\newcommand{\hlkwb}[1]{\textcolor[rgb]{0.69,0.353,0.396}{#1}}%
\newcommand{\hlkwc}[1]{\textcolor[rgb]{0.333,0.667,0.333}{#1}}%
\newcommand{\hlkwd}[1]{\textcolor[rgb]{0.737,0.353,0.396}{\textbf{#1}}}%
\let\hlipl\hlkwb

\usepackage{framed}
\makeatletter
\newenvironment{kframe}{%
 \def\at@end@of@kframe{}%
 \ifinner\ifhmode%
  \def\at@end@of@kframe{\end{minipage}}%
  \begin{minipage}{\columnwidth}%
 \fi\fi%
 \def\FrameCommand##1{\hskip\@totalleftmargin \hskip-\fboxsep
 \colorbox{shadecolor}{##1}\hskip-\fboxsep
     % There is no \\@totalrightmargin, so:
     \hskip-\linewidth \hskip-\@totalleftmargin \hskip\columnwidth}%
 \MakeFramed {\advance\hsize-\width
   \@totalleftmargin\z@ \linewidth\hsize
   \@setminipage}}%
 {\par\unskip\endMakeFramed%
 \at@end@of@kframe}
\makeatother

\definecolor{shadecolor}{rgb}{.97, .97, .97}
\definecolor{messagecolor}{rgb}{0, 0, 0}
\definecolor{warningcolor}{rgb}{1, 0, 1}
\definecolor{errorcolor}{rgb}{1, 0, 0}
\newenvironment{knitrout}{}{} % an empty environment to be redefined in TeX

\usepackage{alltt}
\IfFileExists{upquote.sty}{\usepackage{upquote}}{}
\begin{document}

\title{Sense and Sensibility Wordcloud}
\author{Andrew Innes}
\maketitle

\begin{abstract}
In this article we construct a wordcloud, using the tidytext R package, for Jane Austen's Sense and Sensibility.

\end{abstract}

\textit{Sense and Sensibility} is a novel by Jane Austen, published in 1811.  Below we construct a wordcloud for the most common words appearing in the novel.

\section{The Jane Austen Package}
There is a relatively new package for R, janeaustenr, that gives one access to all of the novels written by Jane Austen.  One first has to install this package and bring it in with library.  You may then call the following function and store the result.  The result will be a data frame.

\begin{knitrout}
\definecolor{shadecolor}{rgb}{0.969, 0.969, 0.969}\color{fgcolor}\begin{kframe}
\begin{alltt}
\hlkwd{library}\hlstd{(janeaustenr)}
\end{alltt}


{\ttfamily\noindent\color{warningcolor}{\#\# Warning: package 'janeaustenr' was built under R version 3.4.2}}\begin{alltt}
\hlstd{sns}\hlkwb{<-}\hlkwd{austen_books}\hlstd{()}
\end{alltt}
\end{kframe}
\end{knitrout}

This dataframe has two columns, one for each line in Austen's novels, and one for indicating which book the line is from.  Let's first filter, using dplyr, so that we have only the lines from Sense and Sensibility:

\begin{knitrout}
\definecolor{shadecolor}{rgb}{0.969, 0.969, 0.969}\color{fgcolor}\begin{kframe}
\begin{alltt}
\hlkwd{library}\hlstd{(dplyr)}
\hlstd{sns}\hlkwb{<-}\hlstd{sns}\hlopt
  \hlkwd{filter}\hlstd{(book} \hlopt{==} \hlstr{'Sense & Sensibility'}\hlstd{)}
\hlkwd{head}\hlstd{(sns)}
\end{alltt}
\begin{verbatim}
## # A tibble: 6 x 2
##                    text                book
##                   <chr>              <fctr>
## 1 SENSE AND SENSIBILITY Sense & Sensibility
## 2                       Sense & Sensibility
## 3        by Jane Austen Sense & Sensibility
## 4                       Sense & Sensibility
## 5                (1811) Sense & Sensibility
## 6                       Sense & Sensibility
\end{verbatim}
\end{kframe}
\end{knitrout}

\noindent Now we are ready for some data cleaning.

\section{Some Data Cleaning}

We would like to remove all of the `Chapter' lines. We can use dplyr again, along with the package stringr.

\begin{knitrout}
\definecolor{shadecolor}{rgb}{0.969, 0.969, 0.969}\color{fgcolor}\begin{kframe}
\begin{alltt}
\hlkwd{library}\hlstd{(stringr)}
\end{alltt}


{\ttfamily\noindent\color{warningcolor}{\#\# Warning: package 'stringr' was built under R version 3.4.2}}\begin{alltt}
\hlstd{sns}\hlkwb{<-}\hlstd{sns}\hlopt
  \hlkwd{filter}\hlstd{(}\hlopt{!}\hlkwd{str_detect}\hlstd{(sns}\hlopt{$}\hlstd{text,}\hlstr{'^CHAPTER'}\hlstd{))}
\end{alltt}
\end{kframe}
\end{knitrout}

Next, we would like to remove the front matter.  By inspection, we have determined that the front matter ends on line 11.  Therefore we can redefine sns to begin on line 12:

\begin{knitrout}
\definecolor{shadecolor}{rgb}{0.969, 0.969, 0.969}\color{fgcolor}\begin{kframe}
\begin{alltt}
\hlstd{sns}\hlkwb{<-}\hlstd{sns[}\hlnum{12}\hlopt{:}\hlnum{12574}\hlstd{,]}
\end{alltt}
\end{kframe}
\end{knitrout}

\section{The Wordcloud}
To make the wordcloud, we first have to break up the lines into words.  We can use a function from the tidytext package for this:

\begin{knitrout}
\definecolor{shadecolor}{rgb}{0.969, 0.969, 0.969}\color{fgcolor}\begin{kframe}
\begin{alltt}
\hlkwd{library}\hlstd{(tidytext)}
\end{alltt}


{\ttfamily\noindent\color{warningcolor}{\#\# Warning: package 'tidytext' was built under R version 3.4.2}}\begin{alltt}
\hlstd{words_df}\hlkwb{<-}\hlstd{sns}\hlopt
  \hlkwd{unnest_tokens}\hlstd{(word,text)}

\hlstd{words_df}
\end{alltt}
\begin{verbatim}
## # A tibble: 119,850 x 2
##                   book     word
##                 <fctr>    <chr>
##  1 Sense & Sensibility      the
##  2 Sense & Sensibility   family
##  3 Sense & Sensibility       of
##  4 Sense & Sensibility dashwood
##  5 Sense & Sensibility      had
##  6 Sense & Sensibility     long
##  7 Sense & Sensibility     been
##  8 Sense & Sensibility  settled
##  9 Sense & Sensibility       in
## 10 Sense & Sensibility   sussex
## # ... with 119,840 more rows
\end{verbatim}
\end{kframe}
\end{knitrout}

We can remove the common, unimportant words with the stop\_words data frame and some dplyr:

\begin{knitrout}
\definecolor{shadecolor}{rgb}{0.969, 0.969, 0.969}\color{fgcolor}\begin{kframe}
\begin{alltt}
\hlstd{words_df}\hlkwb{<-}\hlstd{words_df}\hlopt
  \hlkwd{filter}\hlstd{(}\hlopt{!}\hlstd{(word} \hlopt \hlstd{stop_words}\hlopt{$}\hlstd{word))}

\hlstd{words_df}
\end{alltt}
\begin{verbatim}
## # A tibble: 36,225 x 2
##                   book      word
##                 <fctr>     <chr>
##  1 Sense & Sensibility    family
##  2 Sense & Sensibility  dashwood
##  3 Sense & Sensibility   settled
##  4 Sense & Sensibility    sussex
##  5 Sense & Sensibility    estate
##  6 Sense & Sensibility residence
##  7 Sense & Sensibility   norland
##  8 Sense & Sensibility      park
##  9 Sense & Sensibility    centre
## 10 Sense & Sensibility  property
## # ... with 36,215 more rows
\end{verbatim}
\end{kframe}
\end{knitrout}

\end{document}
